%%%%%%%%%%%%%%%%%%%%%%%%%%%%%%%%%%%%%%%%%%%%%%%%%%%%%%%%%%%%%%%%%%%%%%%%%%%%%%%%
%%%%%%%%%%%%%%%%%%%%%%%%%%%%%%%%%%%%%%%%%%%%%%%%%%%%%%%%%%%%%%%%%%%%%%%%%%%%%%%%
\addchap{Abstract}

The non-holonomic omnidirectional mobile robot has great potential in the industry market. The platform is equiped with 4 conventional centered wheels, and thus endued with large loading capability that is comparable with differential drive structures. In the mean time, the omnidirectional property makes it outperform differential drive structures in terms of flexibility. On the other hand, the holonomic omnidirectional counterparties has better maneuver flexibility. But the clastor or Swedish wheel they equipped can not hold large amount of loading and is more vulnerable to uneven ground. All these factors make non-holonomic omnidirectional structure a competitive candidate for industry use. 


However the control of this structure is still a complex task. This kind of robot only have 1 degree of mobility(DoF) which means they can not change their task space velocity instantly. Such constraint limit us from making full use of it's omnidirectional property. The existing solutions are focusing on pre-defined trajectory, but connected to navigation is a pre-request for it to be intelligent. Autonomous navigation for omnidirectional robot can generate trajectory that is inconsistent with time, especially for obstacle avoidance. Which is still a big challenge for the platform control.

The scope of this work is on the low-level controller which can take inconsistent trajectory and make full use of the maneuverability to acheive it. The low-level controller takes the task space velocity/acceleration command from global planner as input, and output the joint reference signal to joint motors. Our work assumes the existence of an omnidirectional global planner and a well-tuned motor dynamic controller.
%Since the augmented linear model predictive control scheme is linear the computational effort is comparable to conventional linear model predictive control.