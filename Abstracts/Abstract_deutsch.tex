%%%%%%%%%%%%%%%%%%%%%%%%%%%%%%%%%%%%%%%%%%%%%%%%%%%%%%%%%%%%%%%%%%%%%%%%%%%%%%%%
%%%%%%%%%%%%%%%%%%%%%%%%%%%%%%%%%%%%%%%%%%%%%%%%%%%%%%%%%%%%%%%%%%%%%%%%%%%%%%%%
\addchap{Deutsche Kurzfassung}

Die Regelung der Gebäudeklimatik hat zwei große Vorteile.
Erstens erlaubt sie das anpassen der Temperatur und der Luftfeuchtigkeit an die gewünschten Werte.
Dies kann beispielsweise in Bürogebäuden geschehen um angenehme Arbeitsbedingungen zu erzeugen oder in einem Gewächshaus um den Pflanzenwachstum zu maximieren.
Zweitens ermöglicht die Klimaregelung die Minimierung der Energieverbrauchs und -kosten.

Dabei ist das gleichzeitige Regeln eines Gebäudes und die Kostenminimierung schwer zu erreichen mit konventionellen Regelungsmethoden.
Modellprädiktive Regelung kann dieses Ziel erreichen aufgrund ihrer Fähigkeit Prädiktionen wie die Wettervorhersage in ihre Berechnungen einzubeziehen.
Desweiteren berücksichtigt modellprädiktive Regelung Bedingungen wie die Grenzwerte der Stellgrößen.

Mit nichtlinearer modellprädiktiver Regelung werden oft die besten Ergebnisse erzielt, weil die Klimadynamik nichtlinear und hochkomplex ist.
Da lineare modellprädiktive Regelung die nichtlinearen Anteile vernachlässigt ist ihre Genauigkeit geringer.
Die geringere Präzision vermindert die Kosteneinsparungen führt zu vermehrtem Verletzen der Bedingungen im Vergleich zu nichtlinearer modellprädiktiver Regelung.
Jedoch erfordert das nichtlineare Verfahren mehr Rechenleistung als das Lineare.
Desweiteren ist das Ableiten eines nichtlinearen Modells zeitaufwendig und erfordert weitreichende Fachkenntnisse.

Deshalb wird eine mit Gaußschen Prozessen erweiterte lineare modellprädiktive Regelung vorgestellt.
Die Gaußschen Prozesse werden verwendet um die vernachlässigten Nichtlinearitäten abzubilden und somit die Genaugkeit der Prädiktionen zu erhöhen im Vergleich zu konventioneller modellprädiktiver Regelung.
Die Gaußschen Prozesse werden als Parameter im linearen Prädiktionsmodell eingefügt.
Dadurch bleiben die Vorteile der linearen modellprädiktiven Reglung erhalten bei gleichzeitiger Verminderung der Nachteile.

In dieser Arbeit wurde gezeigt, dass die Kombination von linearer modellprädiktiver Regelung und Gaußschen Prozessen vergleichbare Ergebnisse erzielt bezüglich Kostenminimierung und Berücksichtigung der Bedingungen wie  nichtlineare modellprädiktive Regelung.
%Die erweiterte lineare modellprädiktive Regelung erfordert eine vergleichbare Rechenleistung wie konventionelle lineare modellprädiktive Regelung.