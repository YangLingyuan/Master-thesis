%%%%%%%%%%%%%%%%%%%%%%%%%%%%%%%%%%%%%%%%%%%%%%%%%%%%%%%%%%%%%%%%%
%2345678901234567890123456789012345678901234567890123456789012345
%        1         2         3         4         5         6     
\chapter{Conclusion}
\label{cha:conclusion}

% state problem shortly
Building climate control is important due to provide the optimal conditions, may it be for working inside an office building or for crop growth.
Moreover, building climate control gives the possibility to reduce the costs for operating buildings and to reduce the consumption of energy.

For building climate control the best results are often achieved with nonlinear model predictive control because the climate dynamic is highly complex and nonlinear.
Since obtaining a nonlinear prediction model is an time-consuming task and requires expertise, the work presented an augmented linear model predictive control scheme providing similar results to nonlinear model predictive control.
The error due to neglecting the nonlinear dynamics with a linear prediction model is estimated with Gaussian processes and included in the computations as a parameter.\par\medskip

% contribution
It was shown using the example of a greenhouse that error estimation with Gaussian processes improves the quality for the control significantly.
For set-point tracking the root mean square error was reduced by nearly one order of magnitude.
Applying the augmentation in an economic model predictive control scheme resulted in lower costs for operating the greenhouse while providing better conditions for the crop growth.
Thus, the performance of nonlinear model predictive control and linear model predictive control with error estimation via Gaussian processes are comparable.
Furthermore, the advantages of economic model predictive control were demonstrated.
Controlling the greenhouse climate with economic model predictive control resulted in significant cost savings compared to set-point tracking model predictive control.\par\medskip

% future work
One possible improvement for the by Gaussian processes extended linear model predictive control scheme is the estimation of periodic occurring errors as it was shown in \cite{Klenske.2016}.
Since disturbance variables like the solar radiation are linked to the day-night-cycle a Gaussian process with periodic kernel could model this error.
On-line optimization of the hyperparameters would allow adaption to the current weather situation.

Another direction for further developments is the application of the augmented linear model predictive control scheme on embedded systems.
This is possible due to the low computational requirements compared to nonlinear model predictive control.
For the implementation \textbf{\textmu AO-MPC}, a code generation software package for linear model predictive control, is proposed.
For more details see \cite{Zometa.2013}.

Finally, \cite{Plate.1999} presents a more systematic approach to obtain Gaussian process models.
The influence of all input dimensions and the relations between those can be gradually examined.
By dropping input dimensions and relations with little impact the size of the Gaussian process model is decreased and the interpretability is increased. 
Thus, the computational expense of the Gaussian process models is reduced.