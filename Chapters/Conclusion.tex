%%%%%%%%%%%%%%%%%%%%%%%%%%%%%%%%%%%%%%%%%%%%%%%%%%%%%%%%%%%%%%%%%
%2345678901234567890123456789012345678901234567890123456789012345
%        1         2         3         4         5         6     
\chapter{Conclusion}
\label{cha:conclusion}

% state problem shortly
The non-holonomic omnidirectional mobile robot has great potential in the industry market. Yet the control of it is still a complex task to be solved. The constraint on degree of mobility limit us from making full use of it's omnidirectional property. The existing solutions are focusing on pre-defined trajectory, but connected to navigation is a pre-request for it to be intelligent. Autonomous navigation for omnidirectional robot can generate trajectory that is inconsistent with time, especially for obstacle avoidance. Which is still a big challenge for the platform control.




% contribution
In this paper, we review the existing solutions and bring the idea of ICR optimization by \cref{sorour2016motion}. Targeting the drawback of such idea, we propose an approximation based inverse kinematic control method as complementary. Our final solution combine the two control logic and come up with a switching logic to choose the control method that is most appliable for the current state of the platform. From the previous chapter \cref{cha:experiment}, we can see that our controller basically met the design objective. When inconsisteny happens, the smoothed velocity command will be executed instead of the original reference velocity. Which makes it capable of handling the inconsistency trajectory. \par\medskip

% future work
There are a few control problems left unsolved in our paper.

First problem is kinematic singularity. This term is used to refer the case when the ICR move close to the wheel heel point $ICR=[h_{xi},h_{xi}]$. In such situation, the steering angle of the wheel ICR is on is not defined. These singularities is ignored in this paper as it need significant effort to work on. Recent work has proposed some potential solutions \cref{sorour2019complementary}.

Second problem is that the optimization based control logic is not robust. The pre-request for the optimization algorithm to find the optimal point is that the feasible region is not empty. But in practice, the drives of motor have error and the ICR is ill-defined as we mentioned in \cref{fig:illDefinedICR}. The feasible region is defined as the intersection of four steering feasible regions. which is calculated based on the current steering angle. Some times the error is large enough that no  intersection exist anymore, and the optimization algorithm will fail. Some machanism need to be implemented to specially treat such situation to make this controller robust.