%%%%%%%%%%%%%%%%%%%%%%%%%%%%%%%%%%%%%%%%%%%%%%%%%%%%%%%%%%%%%%%%%
%2345678901234567890123456789012345678901234567890123456789012345
%        1         2         3         4         5         6     
\tdplotsetmaincoords{70}{110}     
\chapter{Approximation based Inverse Kinematic control}
\label{cha:inverseKinematics}
Due to the non-linear natural of the kinematic system, it is very hard to apply simple controllers on it. But as mentioned before, in some regions the kinematic system can be linearized without much error \cref{subsec:switching}. Based on such fact, we proposed a strategy that linearize the kinematic model about the current platform state at each control instance by first order approximation. Thus we gain the linear system property and simply scale down the input of the system to limit the output joint space commands so that they don't violate acceleration limits.

%%%%%%%%%%%%%%%%%%%%%%%%%%%%%%%%%%%%%%%%%%%%%%%%%%%%%%%%%%%%%%%%%%%%%%%%%%%%%%%%%%%%%%%%%%%%%%%%%%%%%%%%%%%%%%%%%%%%%%%%%%%%%%%%%%%%%%%%%%%%%%%%%%%%%%%%%%%%%%%%%%%%%%%%%%%%%%%%%%%%%%%%%%%%%

\section{First order approximation}
\label{sec:firstOrderApp}
The kinematic equations used to calculate joint commands are \cref{eq:beta},\cref{eq:betaDot} and \cref{eq:phi}. As we are focusing on the acceleration limits rather than speed limits, \cref{eq:beta} is ignored here.

To distinguish the notions for different time step, we use $\dot{\xi}^n$ to indicate the current state of the platform, while $\dot{\xi}^{n-1},\dot{\xi}^{n+1}$ corresponding to the previous and next time step command.
Through Taylor Series expansions, we linearize \cref{eq:betaDot} and \cref{eq:phi} about the current platform status $\beta, \dot{\xi}^{n}$

\begin{equation}\label{eq:firstOrderApp}
\begin{split}
\dot{\beta}_i^{n+1}&=\dot{\beta}_i^n + \Delta\dot{\beta}_i=f_{1i}(\dot{\xi}^n)(\Delta\ddot{\xi}+\ddot{\xi}^n)\\
\dot{\phi}_i^{n+1}&=\dot{\phi}_i^n + \Delta\dot{\phi}_i=f_{2i}(\beta^n)(\Delta\dot{\xi}+\dot{\xi}^n)
\end{split}
\end{equation}
Assuming the acceleration applied during a control cycle is static, from the linearized approximation, we can extract the relationship between the joint jerk and task space command:
\begin{equation}\label{eq:deltaRelation}
    \begin{split}
        \ddot{\beta}^{n+1}&=\Delta\dot{\beta}_i/\Delta T=\frac{1}{\Delta T}f_{1i}(\dot{\xi}^n)\Delta\ddot{\xi}\\
        \ddot{\phi}^{n+1}&=\Delta\dot{\phi}_i/\Delta T=\frac{1}{\Delta T}f_{2i}(\beta^n)\Delta\dot{\xi}
    \end{split}
\end{equation}

The term $\Delta\dot{\xi}=\dot{\xi}^{n+1}-\dot{\xi}^n$ is a vector indicate the amount of changes of the task space velocity command, which is illustrate in \cref{fig:deltaXi}. And $\Delta\ddot{\xi}=\ddot{\xi}^{n+1}-\ddot{\xi}^n$ is a vector indicate the amount of changes of the task space acceleration command. In practice it is obtained by differentiation.
\begin{figure}\label{fig:deltaXi}
    \centering
    \begin{tikzpicture}[scale=1.5,tdplot_main_coords]      
        \draw[thick,black,->] (0,0,0) -- (3,0,0) node[anchor=north east]{$x$};       
        \draw[thick,black,->] (0,0,0) -- (0,3,0) node[anchor=north west]{$y$};       
        \draw[thick,black,->] (0,0,0) -- (0,0,3) node[anchor=south]{$\theta$};        
        \draw[black,->] (0,0,0) -- (2,2,3) node[anchor=north west]{$\dot{\xi}^n$};
        \draw[black,->] (0,0,0) -- (3,1,2) node[anchor=north east]{$\dot{\xi}^{n+1}$};
        \draw[thick,blue,->] (2,2,3) -- (3,1,2) node[anchor= west]{$\Delta\dot{\xi}$};
\end{tikzpicture}
    \caption{$\Delta\dot{\xi}$}
    \label{fig:deltaXi}
\end{figure}  

%%%%%%%%%%%%%%%%%%%%%%%%%%%%%%%%%%%%%%%%%%%%%%%%%%%%%%%%%%%%%%%%%%%%%%%%%%%%%%%%%%%%%%%%%%%%%%%%%%%%%%%%%%%%%%%%%%%%%%%%%%%%%%%%%%%%%%%%%%%%%%%%%%%%%%%%%%%%%%%%%%%%%%%%%%%%%%%%%%%%%%%%%%%%%%%%

\section{Scaling the command}
\label{sec:Scaling}
Given the relationship we got from previous section, we can calculate the required joint accelerations through \cref{eq:deltaRelation} every time new commands $\dot{\xi},\ddot{\xi}$ arrives. When inconsistency happens, the required joint acceleration may violates the maximum joint capabilities. To deal with such situation, we need to scale down the $\Delta\dot{\xi},\Delta\ddot{\xi}$ and recalculate the executed command $\dot{\xi}_{exe},\ddot{\xi}_{exe}$ . Thus we introduce two scaling factor.

\begin{equation}\label{eq:scalingFactor}
    \begin{split}
        s_{\beta}&=\mid\frac{\ddot{\beta}_{max}}{\ddot{\beta}^{n+1}}\mid \qquad if \mid\ddot{\beta}^{n+1}\mid>\mid\ddot{\beta}_{max}\mid \\
        s_{\phi}&=\mid\frac{\ddot{\phi}_{max}}{\ddot{\phi}^{n+1}}\mid \qquad if \mid\ddot{\phi}^{n+1}\mid>\mid\ddot{\phi}_{max}\mid
    \end{split}
\end{equation}


These two scaling factor corresponding to two different situations with 2 ways of scaling. In practice they may conflict with each other, for the approximation based inverse kinematic control, we only implement the steering scaling in section \cref{subsec:scaleBeta}.

%%%%%%%%%%%%%%%%%%%%%%%%%%%%%%%%%%%%%%%%%%%%%%%%%%%%%%%%%%%%%%%%%%%%
\subsection{Violating steering acceleration limits}\label{subsec:scaleBeta}
In the begining of each control cycle, the first equation of\cref{eq:deltaRelation} will be used to determine if we will violate steering acceleration constraint. If so, the following process will be conducted to scale down the command.

\begin{itemize}
    \item Calculate the scaling factor $s_{\beta}$ through \cref{eq:scalingFactor}
    \item Update the executed task space acceleration $\ddot{\xi}_{exe}=\ddot{\xi}^n+s_{\beta}\Delta\ddot{\xi}$
    \item Based on the updated task space acceleration calculate corresponding task space velocity $\dot{\xi}_{exe}=\dot{\xi}^n+\Delta T\ddot{\xi}_{exe}$
    \item Run the inverse kinematics\cref{eq:beta}, \cref{eq:betaDot} and \cref{eq:phi} with updated $\ddot{\xi}_{exe}, \dot{\xi}_{exe}$ as input.
\end{itemize}

As our method is an approximation, the resulting joint space command may still slightly violate the joint limit. In practice, $\ddot{\beta}_{max}$ is set to be lower than the real joint capability.
%%%%%%%%%%%%%%%%%%%%%%%%%%%%%%%%%%%%%%%%%%%%%%%%%%%%%%%%%%%%%%%%%%%%%%%%%%%%%%%%%%%%%%%%%%%%%%%%%%%%%%%%%%%%%%%%%%%%%%%%%%%%%%%%%%%%%%%%%%%%%%%%%%%%%%%%%%%%%%%%%%%%%%%%%%%%%%%%%%%%%%%%%%%%%%%%
\subsection{Violating driving acceleration limits}
The way we scale down driving speed is as follows:
\begin{itemize}
    \item Calculate the scaling factor $s_{\phi}$ through \cref{eq:scalingFactor} if constraint violation detected
    \item Update the executed task space velocity $\dot{\xi}_{exe}=\dot{\xi}^n+s_{\phi}\Delta\dot{\xi}$
    \item Based on the updated task space velocity calculate corresponding task space acceleration $\ddot{\xi}_{exe}=(\dot{\xi}_{exe}-\ddot{\xi}^n)/\Delta T$
    \item Run the inverse kinematics\cref{eq:beta}, \cref{eq:betaDot} and \cref{eq:phi} with updated $\ddot{\xi}_{exe}, \dot{\xi}_{exe}$ as input.
\end{itemize}


Unlike steering acceleration violation, scaling for driving is applied on $\Delta\dot{\xi}$. These 2 strategy will potentially conflict with each other so that they cannot be applied together.
On Approximation based Inverse Kinematic control, we only implement the scaling for steering acceleration constraint. That is because the driving motor is equiped with velocity PID. Which makes it much more robust to sudden change of the reference signal. The scaling for driving acceleration constraint is then used in \cref{sec:ICRoptimization}.