%%%%%%%%%%%%%%%%%%%%%%%%%%%%%%%%%%%%%%%%%%%%%%%%%%%%%%%%%%%%%%%%%
%2345678901234567890123456789012345678901234567890123456789012345
%        1         2         3         4         5         6     
\chapter{Introduction}
\label{cha:introduction}

Mobile robot has been widely used in the industry, like logistics and palletizing, to decrease the dependence on manpower. Among various structrues of mobile robot, omnidirectinal ones are gaining more and more 
importance due to their high maneuverability in the task space, which makes them more flexible and adaptive in complex envirement. The omnidirectinal mobile robots in the market now are mostly equiped with 
Swedish wheel or claster wheel, these kinds of structures are called holonomic omnidirectinal as they have 3 degree of freedom in the joint space w.r.t 3 degree of freedom in the task space. However, their complex 
wheel structure causes a lot of problem in practice, such as vulnerable to uneven ground and noise etc. To avoid such drawback while make use of the flexibility. We put our sight on the so called non-holonomic 
omnidirectinal mobile robot. 
%%%%%%%%%%%%%%%%%%%%%%%%%%%%%%%%%%%%%%%%%%%%%%%%%%%%%%%%%%%%%%%%%%%%%%%%%%%%%%%%%%%%%%%%%%%%%%%%%%%%%%%%%%%%%%%%%%%%%%%%%%%%%%%%%%%%%%%%%%%%%%%%%%%%%%%%%%%%%%%%%%%%%%%%%%%%%%%%%%%%%%%%%%%%%%%%%
%%%%%%%%%%%%%%%%%%%%%%%%%%%%%%%%%%%%%%%%%%%%%%%%%%%%%%%%%%%%%%%%%%%%%%%%%%%%%%%%%%%%%%%%%%%%%%%%%%%%%%%%%%%%%%%%%%%%%%%%%%%%%%%%%%%%%%%%%%%%%%%%%%%%%%%%%%%%%%%%%%%%%%%%%%%%%%%%%%%%%%%%%%%%%%%%%

Such kind of robots are equipped with fully steerable wheels (four in our case) that can be controled individualy. To control this kind of robots effciently without actuator conflict, Instantaneous Center of 
Rotation(ICR) is always used to moldeling the kinematics. This ICR is defined as being the point in the robot frame that instantaneously does not move in relation to the robot. For our case, the ICR 
corresponds to the point where the zero motion line of each wheel intersect. In practice, we will calculate the desired ICR position/velocity based on the incoming task space command. And coordinate 4 wheels to 
orient to the right position to conduct the task space motion as commanded. Hoever, in practice, the incoming task space velocity command are not smooth w.r.t time, which leads to inconsist jump of ICR and thus
inconsistency of steer angle of wheels. This process of initialization is a main drawback of this kind of structure comparing to holonomic omnidirectinal one. Most study recently are focusing on pre-defined 
trajectories that has been smoothed in the task space\cite{dietrich2011singularity},


%%%%%%%%%%%%%%%%%%%%%%%%%%%%%%%%%%%%%%%%%%%%%%%%%%%%%%%%%%%%%%%%%%%%%%%%%%%%%%%%%%%%%%%%%%%%%%%%%%%%%%%%%%%%%%%%%%%%%%%%%%%%%%%%%%%%%%%%%%%%%%%%%%%%%%%%%%%%%%%%%%%%%%%%%%%%%%%%%%%%%%%%%%%%%%%%%
%%%%%%%%%%%%%%%%%%%%%%%%%%%%%%%%%%%%%%%%%%%%%%%%%%%%%%%%%%%%%%%%%%%%%%%%%%%%%%%%%%%%%%%%%%%%%%%%%%%%%%%%%%%%%%%%%%%%%%%%%%%%%%%%%%%%%%%%%%%%%%%%%%%%%%%%%%%%%%%%%%%%%%%%%%%%%%%%%%%%%%%%%%%%%%%%%
Such a process make this kidn
The objective of this paper is to get rid of this initialization for each velocity command

%%%%%%%%%%%%%%%%%%%%%%%%%%%%%%%%%%%%%%%%%%%%%%%%%%%%%%%%%%%%%%%%%%%%%%%%%%%%%%%%%%%%%%%%%%%%%%%%%%%%%%%%%%%%%%%%%%%%%%%%%%%%%%%%%%%%%%%%%%%%%%%%%%%%%%%%%%%%%%%%%%%%%%%%%%%%%%%%%%%%%%%%%%%%%%%%%
%%%%%%%%%%%%%%%%%%%%%%%%%%%%%%%%%%%%%%%%%%%%%%%%%%%%%%%%%%%%%%%%%%%%%%%%%%%%%%%%%%%%%%%%%%%%%%%%%%%%%%%%%%%%%%%%%%%%%%%%%%%%%%%%%%%%%%%%%%%%%%%%%%%%%%%%%%%%%%%%%%%%%%%%%%%%%%%%%%%%%%%%%%%%%%%%%

The remainder of this thesis is organized as follows.
\cref{cha:greenhousemodel} discusses how to describe a greenhouse system based on (pseudo-) physical equations.
The MPC scheme is illustrated in \cref{cha:mpc} using the example of a linear discrete-time system.
In \cref{cha:gaussianprocesses} Gaussian Processes and their stochastic fundamentals are explained.
\cref{cha:implementation} describes how MPC and GPs are combined in a linear MPC scheme and which tools were used for the implementation.
In the following two chapters the results are discussed.
In both chapters the augmented MPC is compared to conventional linear MPC and nonlinear MPC.
At first, we take a look at the performance respective to set-point tracking in \cref{cha:setpoint}.
Afterwards, in \cref{cha:economic}, an economic MPC approach is analysed and the benefit of economic MPC over set-point tracking MPC is illustrated.
In \cref{cha:conclusion} the outcomes are put into conclusion and directions for future work are provided.