%%%%%%%%%%%%%%%%%%%%%%%%%%%%%%%%%%%%%%%%%%%%%%%%%%%%%%%%%%%%%%%%%
%2345678901234567890123456789012345678901234567890123456789012345
%        1         2         3         4         5         6     
\chapter{Introduction}
\label{cha:introduction}

Building climate control is a challenging task.
The climate of a building is highly complex and nonlinear system.
Using well-known approaches like PID controllers works well for single input single output (SISO) systems.
Whereas the implementation of PID controllers for multiple input multiple output (MIMO) systems like the building climate is tough.

Instead of using classical control approaches model predictive control (MPC) is applied.
MPC is designed to compute the optimal inputs for a MIMO system by predicting the future evolution of the system.
Moreover, MPC directly considers constraints like the limits of actuators.

Due to the nature of the climate the best results for building climate control are achieved using nonlinear prediction models.
One drawback of nonlinear MPC is the high computational expense compared to linear MPC as at every computation step a nonlinear optimization problem has to be solved instead of a linear.
Due to this fact linear MPC is easier to implement on micro-controllers which are usually very cheap and robust against environmental influences.
In warm and moist surroundings like inside a greenhouse installing an embedded system could be a proper solution.

However, using a linear prediction model to control a nonlinear process leads to neglecting the nonlinear dynamics.
Thus, the accuracy of linear MPC is lower than nonlinear MPC.

Hence, in this work a linear MPC scheme is presented which is extended with Gaussian processes (GP).
GPs are a machine-learning approach using stochastic methods to generate a function based on training data.
The GPs are used to estimate the error due to neglecting nonlinear dynamics using a linear prediction model.
Since the GPs are included in the prediction model as a precomputed parameter the optimal control problem stays linear.
Therefore, the new scheme provides a similar accuracy compared to nonlinear MPC while only requiring a computational expense compared to linear MPC.\par\medskip

%%%%%%%%%%%%%%%%%%%%%%%%%%%%%%%%%%%%%%%%%%%%%%%%%%%%%%%%%%%%%%%%%%%%%%%%%%%%%%%%%%%%%%%%%%%%%%%%%%%%%%%%%%%%%%%%%%%%%%%%%%%%%%%%%%%%%%%%%%%%%%%%%%%%%%%%%%%%%%%%%%%%%%%%%%%%%%%%%%%%%%%%%%%%%%%%%
%%%%%%%%%%%%%%%%%%%%%%%%%%%%%%%%%%%%%%%%%%%%%%%%%%%%%%%%%%%%%%%%%%%%%%%%%%%%%%%%%%%%%%%%%%%%%%%%%%%%%%%%%%%%%%%%%%%%%%%%%%%%%%%%%%%%%%%%%%%%%%%%%%%%%%%%%%%%%%%%%%%%%%%%%%%%%%%%%%%%%%%%%%%%%%%%%

This work focuses on two aspects to reach the above mentioned goals.
The first one is deriving an extended linear MPC scheme including GPs.
The works \cite{Kocijan.2003} and \cite{Kocijan.2004} present MPC based on GP prediction models.
The GPs are used for black-box identification of the real process and then included as prediction models.
However, in this work an existing linear model was augmented with GPs.

Works where existing models were extended are presented in \cite{Arahal.2005} and \cite{Klenske.2016}.
While in \cite{Arahal.2005} neuronal networks affecting the inputs are added to a finite impulse response model for simulating a greenhouse, in \cite{Klenske.2016} a GP is added to a linear model for predicting periodically occuring errors.
Since for both mentioned approaches of augmenting a linear model the extensions are directly included into the prediction model they are nonlinear MPC schemes.
Whereas, in this work the GPs are included as a precomputed parameter.
Therefore, the provided extended MPC scheme is linear.\par\medskip

The second aspect of the work regards building climate control which gained lot of attention in the recent years.
A survey is given by \cite{Cigler.2013} through describing the challenges of implementing building climate control.

An economic objective is defined for many approaches where MPC is used for climate control.
MPC approaches maximizing the energy efficiency of building climate control are \cite{Oldewurtel.2010} and \cite{Oldewurtel.2012} which are focusing on the weather forecast and \cite{Zhang.2013} which uses scenario-based MPC.
In this work the weather forecast is used for disturbance rejection in \cref{cha:setpoint} and in \cref{cha:economic} also for minimizing the cost of operating a building.
In \cite{Oldewurtel.2010b} real-time prices are considered to reduce the peak electricity demand, whereas in this work in \cref{cha:economic} the real-time prices for energy are considered to minimize the heating cost for a greenhouse.
%In \cite{Halvgaard.2012} applies economic MPC for building climate control in a smart grid.

Since in this work building climate control is shown using the example of a greenhouse following works were considered.
In \cite{RamirezArias.2005} was shown that MPC improves the efficiency of greenhouse heating systems which is important as heating is the main cost for operating a greenhouse.
Works describing a broad range of issues affecting the control of a greenhouse are \cite{Rodriguez.2015} and \cite{Bakker.1995}.\par\medskip

%%%%%%%%%%%%%%%%%%%%%%%%%%%%%%%%%%%%%%%%%%%%%%%%%%%%%%%%%%%%%%%%%%%%%%%%%%%%%%%%%%%%%%%%%%%%%%%%%%%%%%%%%%%%%%%%%%%%%%%%%%%%%%%%%%%%%%%%%%%%%%%%%%%%%%%%%%%%%%%%%%%%%%%%%%%%%%%%%%%%%%%%%%%%%%%%%
%%%%%%%%%%%%%%%%%%%%%%%%%%%%%%%%%%%%%%%%%%%%%%%%%%%%%%%%%%%%%%%%%%%%%%%%%%%%%%%%%%%%%%%%%%%%%%%%%%%%%%%%%%%%%%%%%%%%%%%%%%%%%%%%%%%%%%%%%%%%%%%%%%%%%%%%%%%%%%%%%%%%%%%%%%%%%%%%%%%%%%%%%%%%%%%%%

The first main contribution of this work is to provide an extended linear MPC scheme.
In comparison to conventional linear MPC the extended scheme includes in the prediction model a term estimating the error caused by the linearisation.
This error term is described by GPs and models the neglected nonlinear dynamics.

The second main contribution is the implementation of the provided MPC scheme in building climate control.
It is shown that the extended MPC scheme provides similar results compared to nonlinear MPC and outperforms conventional linear MPC.
This was verified for set-point tracking MPC of a greenhouse climate.
%Nonlinear MPC was defined as the benchmark since it used a perfect prediction model.
Moreover, applying the new scheme in economic MPC led to significant reductions of the operating costs compared to linear MPC.\par\medskip

%%%%%%%%%%%%%%%%%%%%%%%%%%%%%%%%%%%%%%%%%%%%%%%%%%%%%%%%%%%%%%%%%%%%%%%%%%%%%%%%%%%%%%%%%%%%%%%%%%%%%%%%%%%%%%%%%%%%%%%%%%%%%%%%%%%%%%%%%%%%%%%%%%%%%%%%%%%%%%%%%%%%%%%%%%%%%%%%%%%%%%%%%%%%%%%%%
%%%%%%%%%%%%%%%%%%%%%%%%%%%%%%%%%%%%%%%%%%%%%%%%%%%%%%%%%%%%%%%%%%%%%%%%%%%%%%%%%%%%%%%%%%%%%%%%%%%%%%%%%%%%%%%%%%%%%%%%%%%%%%%%%%%%%%%%%%%%%%%%%%%%%%%%%%%%%%%%%%%%%%%%%%%%%%%%%%%%%%%%%%%%%%%%%

The remainder of this thesis is organized as follows.
\cref{cha:greenhousemodel} discusses how to describe a greenhouse system based on (pseudo-) physical equations.
The MPC scheme is illustrated in \cref{cha:mpc} using the example of a linear discrete-time system.
In \cref{cha:gaussianprocesses} Gaussian Processes and their stochastic fundamentals are explained.
\cref{cha:implementation} describes how MPC and GPs are combined in a linear MPC scheme and which tools were used for the implementation.
In the following two chapters the results are discussed.
In both chapters the augmented MPC is compared to conventional linear MPC and nonlinear MPC.
At first, we take a look at the performance respective to set-point tracking in \cref{cha:setpoint}.
Afterwards, in \cref{cha:economic}, an economic MPC approach is analysed and the benefit of economic MPC over set-point tracking MPC is illustrated.
In \cref{cha:conclusion} the outcomes are put into conclusion and directions for future work are provided.