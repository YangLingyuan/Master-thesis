%%%%%%%%%%%%%%%%%%%%%%%%%%%%%%%%%%%%%%%%%%%%%%%%%%%%%%%%%%%%%%%%%
%2345678901234567890123456789012345678901234567890123456789012345
%        1         2         3         4         5         6     
\chapter{ICR optimization control}
\label{cha:ICR}
The ICR optimization controller transform the control problem from task space into a new ICR space. By projecting the task space command $\dot{\xi},\ddot{\xi}$ into ICR, as well as estimate the current ICR by joint 
state feed back from encoder. We get the reference signal $ICR_{ref}$, $\dot{ICR_{ref}}$ as well as the feedback $ICR_{cur}$, so that we form the control problem as an error driven proportional control problem, 
from which we generate the pre-processed control signal$\Tilde{ICR_{ref}} , \Tilde{\dot{ICR_{ref}}}$. They will be fed to the optimization block to check if the command is violating the joint limits and if so, 
find the optimal command with respect to the reference. Finally, calculate the joint command $\beta,\dot{\beta}$ based on the ICR. The original task space command $\dot{\xi}$  will be projected on the null 
space of this configuration, to obtain a feasible control command $\dot{\xi}_{ref}$, and then the reference wheel driving speed$\dot{\phi}$ is calculated based on $\dot{\xi}_{ref}$ and joint configuration 
$\beta,\dot{\beta}$.
%%%%%%%%%%%%%%%%%%%%%%%%%%%%%%%%%%%%%%%%%%%%%%%%%%%%%%%%%%%%%%%%%%%%%%%%%%%%%%%%%%%%%%%%%%%%%%%%%%%
%%%%%%%%%%%%%%%%%%%%%%%%%%%%%%%%%%%%%%%%%%%%%%%%%%%%%%%%%%%%%%%%%%%%%%%%%%%%%%%%%%%%%%%%%%%%%%%%%%%
%%%%%%%%%%%%%%%%%%%%%%%%%%%%%%%%%%%%%%%%%%%%%%%%%%%%%%%%%%%%%%%%%%%%%%%%%%%%%%%%%%%%%%%%%%%%%%%%%%%
\section{Error driven proportional ICR velocity control}\label{sec:ICRvelocityControl}
The objective of introducing an error driven proportional control is to smooth the reference signal, so that the model do not use the maximum acceleration all the time when discontinuity happens. Such a strategy 
would help the robot avoiding stiff movements.

%%%%%%%%%%%%%%%%%%%%%%%%%%%%%%%%%%%%%%%%%%%%%%%%%%%%%%%%%%%%%%%%%%%%%%%%%%%%%%%%%%%%%%%%%%%%%%%%%%%
%%%%%%%%%%%%%%%%%%%%%%%%%%%%%%%%%%%%%%%%%%%%%%%%%%%%%%%%%%%%%%%%%%%%%%%%%%%%%%%%%%%%%%%%%%%%%%%%%%%
\subsection{ICR calculation}
As we introduced before, the ICR, by definition, is the point where the zero motion line of each wheel intersect. But that only applies for estimating the current state $ICR_{curr}$, we use different equation to 
calculate the reference $ICR_{ref}$. Calculation of reference and current ICR position is the first step for the whole control strategy.

%%%%%%%%%%%%%%%%%%%%%%%%%%%%%%%%%%%%%%%%%%%%%%%%%%%%%%%%%%%%%%%%%
\subsubsection{Reference ICR}
The term $ICR_{ref}$ denote the ICR point corresponding to the reference task space velocity command $\dot{\xi}$. For convenience, we also introduce the term $\dot{ICR_{ref}}$ to evaluate the velocity of $ICR_{ref}$ 
which corresponding to $\dot{\xi}$ and $\ddot{\xi}$.
When we receive a task space command $\dot{\xi}=[\dot{x},\dot{y},\dot{\theta}]$, the corresponding $ICR_{ref}=[X_{ref},Y_{ref}]$ could be calculate by equation \cref{eq:ICR_ref}
%%%%%%%%%%%%%%%%%%%%%%%%%%%%%%%%%%
\begin{figure}[t]
	\begin{center}
	\resizebox{10cm}{!}
    {
		\begin{tikzpicture}
			\pic at(0,0) {platform};
			\pic[rotate=61] at(2.63,2.63) {wheelFrame={1}};
			\pic[rotate=11.2] at(2.63,-2.63) {wheelFrame={2}};
			\pic[rotate=-80.3] at(-2.63,2.63) {wheelFrame={4}};
			\pic[rotate=-34] at(-2.63,-2.63) {wheelFrame={3}};		
			\filldraw [gray] (1.4,3.3) circle (2pt) node[above,text=red]{ICR};
			\draw [rotate=-90, thick, ->] (0,0) -- (1.4,3.3) node[above,text=red]{$\dot{\xi}^{ref}$};
			\draw (2.63,2.63) -- (1.4,3.3);
			\draw (2.63,-2.63) -- (1.4,3.3);
			\draw (-2.63,-2.63) -- (1.4,3.3);
			\draw (-2.63,2.63) -- (1.4,3.3);
		\end{tikzpicture}
		}
	\end{center}
	\caption{Reference ICR}
\end{figure}
%%%%%%%%%%%%%%%%%%%%%%%%%%%%%%%%%%
\begin{equation}\label{eq:ICR_ref}
    \begin{split}
        ICR_{ref}=[\dot{x}/\dot{\theta},\dot{y}/\dot{\theta}]
    \end{split}
\end{equation}
And the $\dot{ICR_{ref}}$ is derived by differentiate \cref{eq:ICR_ref} on time.
%%%%%%%%%%%%%%%%%%%%%%%%%%%%%%%%%%
\begin{equation}\label{eq:ICRdot_ref}
    \begin{split}
        \dot{ICR_{ref}}=[\frac{-\ddot{y}\dot{\theta}+\dot{y}\ddot{\theta}}{\dot{\theta}^2},\frac{\ddot{x}\dot{\theta}-\dot{x}\ddot{\theta}}{\dot{\theta}^2}]
    \end{split}
\end{equation}
But these original equations will fail on evaluating pure translation command $\dot{\theta}=0$ or standing still command $\dot{\xi}=[0, 0, 0]$. So we introduce a small modification to work around
%%%%%%%%%%%%%%%%%%%%%%%%%%%%%%%%%%
\begin{equation}\label{eq:ICR_refModified}
    \begin{split}
		 ICR_{ref}&=[\frac{\dot{x}}{\dot{\theta}+sign(\dot{\theta})*\delta} ,\frac{\dot{y}}{\dot{\theta}+sign(\dot{\theta})*\delta}] \\
		 \dot{ICR_{ref}}&=[\frac{-\ddot{y}\dot{\theta}+\dot{y}\ddot{\theta}}{\dot{\theta}^2+\delta},\frac{\ddot{x}\dot{\theta}-\dot{x}\ddot{\theta}}{\dot{\theta}^2+\delta}]
    \end{split}
\end{equation}
where $\delta=1e-8$ is a small damping factor to prevent the $ICR_{ref}$ and $\dot{ICR_{ref}}$ goes to undefined. And $sign()$ is a simple sign function to prevent ICR jump from one extreme to another when the $\dot{\theta}$ is very small.

%%%%%%%%%%%%%%%%%%%%%%%%%%%%%%%%%%%%%%%%%%%%%%%%%%%%%%%%%%%%%%%%%%%%%%%
\subsubsection{Current ICR}
To estimate the current ICR, we simply use the definition of it, the intersection of zero motion lines. In ideal circumstance, all four line will intersect at one point which means the kinematic constraints are 
perfectly respected. However, as always, this is not the case in practice. Each pair of wheels can define a ICR, and in the worst case there will be totally 6 current ICR defined. Such case is illustrated in figure 

\begin{figure}[t]\label{fig:illDefinedICR}
	\begin{center}
	\resizebox{10cm}{!}
    {
		\begin{tikzpicture}
			\pic at(0,0) {platform};
			\pic[rotate=64] at(2.63,2.63) {wheelFrame={1}};
			\pic[rotate=11.2] at(2.63,-2.63) {wheelFrame={2}};
			\pic[rotate=-80.3] at(-2.63,2.63) {wheelFrame={4}};
			\pic[rotate=-33] at(-2.63,-2.63) {wheelFrame={3}};		
			\filldraw [gray] (1.36,3.38) circle (1pt) node[above,text=red]{ICR};
			\draw [rotate=-90, thick, ->] (0,0) -- (1.4,3.3) node[above,text=red]{$\dot{\xi}^{ref}$};
			\draw (2.63,2.63)  -- +(154:3);
			\draw (2.63,-2.63)  -- +(101.2:8);
			\draw (-2.63,-2.63) -- +(57:9);
			\draw (-2.63,2.63)  -- +(9.7:5);
		\end{tikzpicture}
		}
	\end{center}
	\caption{Ill-defined ICR}
\end{figure}



To be able to use our control method, we need to estimate the "real" current ICR based on the sensor information. A simple yet effective method is to use least squre to calculate a point that it's distance to the 
4 zero motion lines are minimized\cite{ICRestimation}. We introduce that method into our controller to improve the accuracy and form the problem as \cref{eq:leastSquare}
\begin{equation}\label{eq:leastSquare}
	\begin{split}
		\underset{ICR_{curr}}{minimize} \quad \sum_{i=1}^{4} d(ICR_{curr},z_i)
	\end{split}
\end{equation}
Where $d(ICR_{curr},z_i)=[\frac{(Y_{est} - tan(\beta_i+\frac{\pi}{2})X_{est} - (hy_i-hx_itan(\beta_i+\frac{\pi}{2})))^2}{tan(\beta_i+\frac{\pi}{2})^2+1}]^2$ is the distance between estimated current ICR and the i-th wheel frame's zero motion line($\overrightarrow{y}$ vector of wheel frame). And $hy_i,hx_i$ is i-the wheel's origin position expressed in the body frame. This problem can be solved analytically and the solution is shown in \cref{eq:leastSquareSolution}

\begin{equation}\label{eq:leastSquareSolution}
	\begin{split}
		(X_{est},Y_{est})=(\frac{BE-CD}{AD-B^2},\frac{AE-BC}{AD-B^2})
	\end{split}
\end{equation}
Where $A=\sum_{i=1}^{4}\frac{2tan(\beta_i+\frac{\pi}{2})^2}{tan(\beta_i+\frac{\pi}{2})^2+1}$, $B=\sum_{i=1}^{4}\frac{2tan(\beta_i+\frac{\pi}{2})}{tan(\beta_i+\frac{\pi}{2})^2+1}$, \\
$C=\sum_{i=1}^{4}\frac{2(hy_i-hx_itan(\beta_i+\frac{\pi}{2}))tan(\beta_i+\frac{\pi}{2})}{tan(\beta_i+\frac{\pi}{2})^2+1}$,\\ $D=\sum_{i=1}^{4}\frac{2}{tan(\beta_i+\frac{\pi}{2})^2+1}$, 
$E =\sum_{i=1}^{4}\frac{2(hy_i-hx_itan(\beta_i+\frac{\pi}{2}))}{tan(\beta_i+\frac{\pi}{2})^2+1}$\\

This solution holds whenever the platform is not performing pure translation and $\beta_i$ is the same for all the wheels.



%%%%%%%%%%%%%%%%%%%%%%%%%%%%%%%%%%%%%%%%%%%%%%%%%%%%%%%%%%%%%%%%%%%%%%%%%%%%%%%%%%%%%%%%%%%%%%%%%%%
%%%%%%%%%%%%%%%%%%%%%%%%%%%%%%%%%%%%%%%%%%%%%%%%%%%%%%%%%%%%%%%%%%%%%%%%%%%%%%%%%%%%%%%%%%%%%%%%%%%
\subsection{ICR velocity control}
The ICR velocity controller is implemented to be error driven in order to interpolate ICR position between inconsistency and not making full use of the jerk. It is formulated in the following way.
\begin{equation}
	\begin{split}
		\Tilde{ICR}_{ref}&=\Tilde{\dot{ICR}}_{ref} \delta T\\
		\Tilde{\dot{ICR}}_{ref}&=\dot{ICR}_{ref} + K_pICR_{error}
	\end{split}
\end{equation}
Where $K_p=4$ is a positive scalar proportional control gain and $ICR_{error}=ICR_{ref}-ICR_{curr}$ is the error between reference ICR and current ICR. This controller will generate a smoothed reference ICR $\Tilde{\dot{ICR_{ref}}}$
To be executed in the following steps.



%%%%%%%%%%%%%%%%%%%%%%%%%%%%%%%%%%%%%%%%%%%%%%%%%%%%%%%%%%%%%%%%%%%%%%%%%%%%%%%%%%%%%%%%%%%%%%%%%%%
%%%%%%%%%%%%%%%%%%%%%%%%%%%%%%%%%%%%%%%%%%%%%%%%%%%%%%%%%%%%%%%%%%%%%%%%%%%%%%%%%%%%%%%%%%%%%%%%%%%
%%%%%%%%%%%%%%%%%%%%%%%%%%%%%%%%%%%%%%%%%%%%%%%%%%%%%%%%%%%%%%%%%%%%%%%%%%%%%%%%%%%%%%%%%%%%%%%%%%%

\section{ICR optimization}\label{sec:ICRoptimization}
The main objective of this controller is to respect the joint limits whille try to follow the reference command. When discontinuity happens, the reference ICR will be far away from the current ICR, even if the 
processed $Tilde{ICR}$ still might corresponding to a joint space command that violates the joint limits. In such cased we need to find out the optimal next sample time ICR point $ICR_{next}$ that does not violate 
joint limits while being as close to the reference ICR as possible. This chapter explained how do we form the optimization problem in \cref{subsec:FormOptimization} and which method we use to solve it \cref{subsec:solveQP}
%%%%%%%%%%%%%%%%%%%%%%%%%%%%%%%%%%%%%%%%%%%%%%%%%%%%%%%%%%%%%%%%%%%%%%%%%%%%%%%%%%%%%%%%%%%%%%%%%%%
%%%%%%%%%%%%%%%%%%%%%%%%%%%%%%%%%%%%%%%%%%%%%%%%%%%%%%%%%%%%%%%%%%%%%%%%%%%%%%%%%%%%%%%%%%%%%%%%%%%
\subsection{Form the optimization problem}
\label{subsec:FormOptimization}

Such a problem can be formed as a quadratic programming problem with 8 linear inequality constraints. Each constraint correspond to a acceleration/deceleration limit, which is shown in \cref{eq:QP}
\begin{equation} \label{eq:QP}
	\begin{split}
		\underset{ICR_{next}}{\textbf{minimize}} \qquad &\parallel ICR_{next}-ICR_{curr}\parallel_2^2\\
		\underset{i=1:4}{\textbf{subject} \quad \textbf{to}} \qquad &(-1)^{q_i(min)}A_{i(min)}(ICR_{next}-hi)>=0\\
												  &(-1)^{q_i(max)+1}A_{i(max)}(ICR_{next}-hi)<=0
	\end{split}
\end{equation}


\begin{figure}[!ht]
	\centering
	\begin{tikzpicture}
			\pic at(0,0) {platform};
			\pic[rotate=61] at(2.63,2.63) {wheelFrame={1}};
			\pic[rotate=11.2] at(2.63,-2.63) {wheelFrame={2}};
			\pic[rotate=-80.3] at(-2.63,2.63) {wheelFrame={4}};
			\pic[rotate=-33] at(-2.63,-2.63) {wheelFrame={3}};		
			\filldraw [gray] (1.4,3.3) circle (1pt) node[above,text=red]{ICR};
			%\draw [rotate=-90, thick, ->] (0,0) -- (1.4,3.3) node[above,text=red]{$\dot{\xi}^{ref}$};
			\draw[dashed] (2.63,2.63)  -- (1.4,3.3);
			\draw (2.63,2.63)  -- +(146:3);
			\draw (2.63,2.63)  -- +(156:3);
			%\draw (2.63,-2.63)  -- +(101.2:8);
			%\draw (-2.63,-2.63) -- +(57:9);
			\draw[dashed] (-2.63,2.63)  -- (1.4,3.3);
			\draw (-2.63,2.63)  -- +(14.7:5);
			\draw (-2.63,2.63)  -- +(7.7:5);
	\end{tikzpicture}
	\caption{}
	\label{}
\end{figure}



In \cref{eq:QP}, the symbol $\parallel \parallel_2^2$ denotes the squared Euclidean distance between 2 points.$\beta_{i(max)/(min)}$ denotes the end up steering angle after the i-th wheel applying the positive/negative 
maximum acceleration on the current steering direction for 1 control cycle. And $A_{i(min)/(max)}=[cot(\beta_{i(min)/(max)}),1]$ is the line parameter,  $h_i=[hx_i,hy_i]$ is the i-th wheel heel
(wheel frame origin) point in the body frame and $q_{i(max)/(min)}$ is a sign indicator for each constraint:
\begin{equation}
	q_{i(max)/(min)}=
	\begin{cases}
		0,&if \beta_{i(max)/(min)}\in 1^{st}or4^{th} quadrant\\
		1,&if \beta_{i(max)/(min)}\in 2^{nd}or3^{rd} quadrant\\
	\end{cases}
\end{equation}

To calculate the maximum and minimized steering angle, we first calculate the maximum/minimum steering velocity. Given the 1 control cycle time ($t_s$) we use the following equation:
\begin{equation}\label{eq:betaDotMax}
	\begin{split}
		\dot{\beta}_i(max)=&
		\begin{cases}
		\dot{\beta_i}+\ddot{\beta}_{max}*t_s, &\dot{\beta}_{_i(max)}<\dot{\beta}_{max}\\
		\dot{\beta}_{max}, &otherwise\\
		\end{cases}\\
		\dot{\beta}_i(min)=&
		\begin{cases}
		\dot{\beta_i}-\ddot{\beta}_{max}*t_s, &\dot{\beta}_{_i(min)}>-\dot{\beta}_{max}\\
		-\dot{\beta}_{max}, &otherwise\\
		\end{cases}
	\end{split}
\end{equation}
And then the maximum/minimum steering angle can be calculated as 
\begin{equation}\label{eq:betaMax}
	\begin{split}
		\beta_i(max/min)= \beta_i + \dot{\beta}_{_i(max/min)}*t_s\\
	\end{split}
\end{equation}
%%%%%%%%%%%%%%%%%%%%%%%%%%%%%%%%%%%%%%%%%%%%%%%%%%%%%%%%%%%%%%%%%%%%%%%%%%%%%%%%%%%%%%%%%%%%%%%%%%%
%%%%%%%%%%%%%%%%%%%%%%%%%%%%%%%%%%%%%%%%%%%%%%%%%%%%%%%%%%%%%%%%%%%%%%%%%%%%%%%%%%%%%%%%%%%%%%%%%%%
\subsection{Solving the linear constraint QP}\label{subsec:solveQP}
So far the optimization problem is successfully formed. There are several methods can solve such an inequality constrained QP problem. Interior-point method is brought up in 1990, efficient on solving large-scale problems. Gradient projection method which is most effective solving problem only with bounding constraints. Alternating direction method of multipliers (ADMM) solve problem by breaking them into smaller pieces, which is extremly usful on parallel computation. And Active set method, old yet effective on solving small-sized problems.

Back to our case, our problem is a 2-D QP with 8 linear constraint, strictly small-scale problem. The most suitable method would Active set method. And our test later does prove that the performance of this method meets the real-time requirement.

We first have a brief view of the optimal conditions for inequality constrained quadratic programming. The Lagrangian for this problem is:
\begin{equation}\label{eq:Lagrangian}
	\begin{split}
		\mathcal{L}(x,\lambda)=\frac{1}{2}x^TPx + q^Tx - \sum_{i=1}^8\lambda_i(A_ix^T-b_i)
	\end{split}
\end{equation}
We know that the Karush-Kuhn-Tucker(KKT) conditions are complementary for a optimal solution. By applying KKT conditions on this specific problem, we have the optimal conditions for our QP problem:
\begin{equation}\label{eq:Lagrangian}
	\begin{split}
		Px+q+\sum_{i=1}^8\lambda_i A_i &=0\\
		A_i^Tx-b_i <&=0\\
		\lambda_i >&=0\\
		\lambda_i(A_i^Tx-b_i) &=0\\
	\end{split}
\end{equation}
The main idea of active set method is that the inactive constraints, for which the inequality strictly holds, can be removed without any effect on the solution. And the group of constraints for which equality hold is called active set. The basic steps of Active set method is 
\begin{itemize}
    \item[1] Approximate the optimal active set
    \item[2] Solve equality constrained problem
    \item[3] Update the active set by gradient and Lagrange multiplier
\end{itemize}
As the main focus of our paper is not this algorithm, we implemented an open source Active set solver that is well optimized. So we won't dig deeper in the theory. 
% \begin{equation} \label{eq:QP}
% 	\begin{split}
% 		\underset{ICR_{next}}{\textbf{minimize}} \qquad &\frac{1}{2}x^TPx + q^Tx\\
% 		\underset{i\in\mathcal{A}(x)}{\textbf{subject} \quad \textbf{to}} \qquad &(-1)^{q_i(min)}A_{i(min)}(ICR_{next}-hi)=0\\
% 												  &(-1)^{q_i(max)+1}A_{i(max)}(ICR_{next}-hi)=0
% 	\end{split}
% \end{equation}

% Where $P=[1,1;1,1], q=[X_{curr},Y_{curr}]$ and $\mathcal{A}(x)$ denotes the current Active set.


After solving the optimization problem, we got a sigle ICR point, based on which we can calculate the next time step refernece steering angle $\beta_{next}$.
%and also the $\dot{\beta}_{next}$ by numeric differentiation of $\beta_{next}$.


\section{Driving speed and steering speed}
The product from the above sections \cref{sec:ICRvelocityControl},\cref{sec:ICRoptimization} is the next step ICR point. Which can only be used to compute the corresponding steering angles, but not steering speed$\dot{\beta}$ or driving speed$\dot{\phi}$. That is because when we map the task space velocity $\dot{\xi}$ to ICR, some information is lost. ICR can no longer been mapped back to a unique $\dot{\xi}$ anymore.Thus we need a separate method to calculate the desired $\dot{\beta}$ and $\dot{\phi}$. Here we apply a similar method as in chapter \cref{cha:inverseKinematics}, but the approximation is simplified to be applied on driving rate. 















The scaled task space velocity $\dot{\xi}^{ref(init)}$ we get from above is incompatible with the $ICR_{next}$ from previous calculation. To be able to give a command that respects the kinematic constraint, we simply project the $\dot{\xi}^{ref(init)}$ onto the null space of Kinematic constraints that is updated by $\beta_{next}$. 
The update of Kinematic constraints is conducted through applying $\beta_{next}$ on \cref{eq:phi}

\begin{equation}\label{eq:updatedKinematics}
    \begin{split}
        f_{2i}(\beta_{inext}) &= \frac{1}{r_w}[cos(\beta_{inext}), sin(\beta_{inext}), -h_{yi}cos(\beta_{inext})+h_{xi}sin(\beta_{inext})]
    \end{split}
\end{equation}
After we updated the kinematic constraint, the scaled task space velocity $\dot{\xi}^{ref(init)}$ is projected onto the null space of it through \cref{eq:projectToNullspace}
\begin{equation}\label{eq:projectToNullspace}
    \begin{split}
       \dot{\xi}^{ref} &= (I-f_{2i}^{+}(\beta_{inext})f_{2i}(\beta_{inext}))\dot{\xi}^{ref(init)}
    \end{split}
\end{equation}

Finally, by applying the resulting reference task space velocity $\dot{\xi}^{ref}$ on \cref{eq:betaDot} and \cref{eq:phi} with updated kinematic constraints \cref{eq:updatedKinematics}. We finally got the next step reference joint space command $\beta_{next}, \dot{\beta}_{next}, \dot{\phi}_{next}$.